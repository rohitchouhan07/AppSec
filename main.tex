\documentclass[conference,compsoc]{IEEEtran}
\ifCLASSOPTIONcompsoc
  \usepackage[nocompress]{cite}
\else
  \usepackage{cite}
\fi
\ifCLASSINFOpdf
   \usepackage[pdftex]{graphicx}
\else
\fi
\usepackage{url}


\begin{document}
\title{Centralized scan Factory for Application Security}
\author{
\IEEEauthorblockN{Rohit Chouhan}
\IEEEauthorblockA{Application Security Engineer\\
PepsiCo, TX, US\\
\url{Email: rchouhan@cs.stonybrook.edu}
}
\and
\IEEEauthorblockN{Sera Kim}
\IEEEauthorblockA{Application Security Engineer\\
PepsiCo, TX, US\\
\url{Email: serakim1105@gmail.com}}
\and
\IEEEauthorblockN{Chad Haley}
\IEEEauthorblockA{Application Security Engineer\\
PepsiCo, TX, US\\
\url{Email: chad.haley443@gmail.com}}
}
\maketitle
\begin{abstract}
\end{abstract}
\IEEEpeerreviewmaketitle

\section{Introduction}
Application security (AppSec) in the modern software landscape encompasses a wide range of domains, including static code analysis, API security, edge and container security,
AI security, and web application firewalls (WAFs). Modern applications are complex, distributed systems composed of numerous interconnected components and services. Securing each of
these layers is a daunting task, making Application Security a critical pillar of enterprise security strategy.

A foundational aspect of AppSec is static source code scanning - the process of analyzing source code to detect potential security vulnerabilities before deployment.
Since source code forms the basis of all applications, identifying and remediating security issues early in the development lifecycle is crucial.

Traditionally, static code analysis has been performed using commercial off-the-shelf (COTS) scanners. These tools inspect application source code to uncover programming flaws
that could lead to vulnerabilities such as SQL injection, cross-site scripting (XSS), or insecure deserialization. They also perform software composition analysis (SCA) to
detect vulnerabilities in third-party dependencies and flag license compliance issues, while secrets scanners identify hardcoded credentials within the codebase.

However, effectively operationalizing these scanners at scale presents significant challenges for large organizations. Continuous scanning, correlating results,
and surfacing actionable findings to the respective code owners remain persistent bottlenecks. A common approach is to embed security scans directly within CI/CD pipelines.
While this appears natural and convenient, it introduces several operational limitations. Integrating scanners across thousands of repositories requires close
coordination with multiple DevOps teams, often leading to inconsistencies in scan coverage and configuration drift. Moreover, embedding scans directly in CI/CD pipelines
limits central control, reduces visibility across projects, and introduces dependencies on individual pipelines’ health and configuration.

To address these challenges, we propose the Centralized AppSec Scan Factory, a cloud-native framework designed to automate, orchestrate,
and scale application security scanning across enterprise repositories. The system integrates static application security testing (SAST), software
composition analysis (SCA), and secrets detection into a unified pipeline. By leveraging event-driven orchestration through message queues and on-demand containerized agents,
the platform dynamically allocates compute resources for scanning tasks while maintaining centralized control, visibility, and reporting.
This architecture ensures consistent scanning coverage with minimal manual intervention, optimizing both performance and cost efficiency.

\newpage
\bibliographystyle{IEEEtran}
\bibliography{references}
\end{document}
\typeout{get arXiv to do 4 passes: Label(s) may have changed. Rerun}

