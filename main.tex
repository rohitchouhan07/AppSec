\documentclass[conference,compsoc]{IEEEtran}
\ifCLASSOPTIONcompsoc
  \usepackage[nocompress]{cite}
\else
  \usepackage{cite}
\fi
\ifCLASSINFOpdf
   \usepackage[pdftex]{graphicx}
\else
\fi
\usepackage{url}


\begin{document}
\title{Centralized scan Factory for Application Security}
\author{
\IEEEauthorblockN{Rohit Chouhan}
\IEEEauthorblockA{Application Security Engineer\\
PepsiCo, TX, US\\
\url{Email: rchouhan@cs.stonybrook.edu}
}
\and
\IEEEauthorblockN{Sera Kim}
\IEEEauthorblockA{Application Security Engineer\\
PepsiCo, TX, US\\
\url{Email: serakim1105@gmail.com}}
\and
\IEEEauthorblockN{Chad Haley}
\IEEEauthorblockA{Application Security Engineer\\
PepsiCo, TX, US\\
\url{Email: chad.haley443@gmail.com}}
}
\maketitle
\begin{abstract}
\end{abstract}
\IEEEpeerreviewmaketitle

\section{Introduction}
Application security program covers a large array of practices setup to protect
modern application against cyberattacks. Mention a good introduction of application
security and the different areas like, API cloud security and explain how this
paper focuses on the scanning part, specifically SAST, DAST, SCA, Secrets and
SBOM. How modern organization typically scan their assets and how a centralized
scan factory provided a complete coverage along with other tunable parameters,
and complete control that is not possible with COTS tools.
\newpage
\bibliographystyle{IEEEtran}
\bibliography{references}
\end{document}
\typeout{get arXiv to do 4 passes: Label(s) may have changed. Rerun}

